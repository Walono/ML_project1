\documentclass[10pt,conference,compsocconf]{IEEEtran}

\usepackage{hyperref}
\usepackage{graphicx}	% For figure environment


\begin{document}
\title{Machine Learning - Project 1}

\author{
  Julie Djeffal \\
  Lo\"{i}s Huguenin \\
  Fabien Zellweger \\
  \textit{Ecole Polythecnique Federale de Lausanne - Switzerland}
}

\maketitle

\begin{abstract}
  
\end{abstract}

\section{Introduction}
%Describe your problem and state your contributions.
%You will do exploratory data analysis to understand your dataset and your features, do feature processing and engineering to clean your dataset and extract more meaningful information, implement and use machine learning methods on real data, analyze your model and generate predictions using those methods and report your findings.
The purpose of this project is to predict if a set of features represent a signal or a background noise. 
To do this, two set of data are provided by The Higgs boson machine learning challenge~\cite{challenge} that took place on Kaggle from May to September 2014. The first data set is a train data set, containing a mixture of simulated signal and background events.
$y_{train}$, the output variable is a binary variable that can take the values $'s'$ or $'b'$, depending of if the output is a signal or a background noise. The 30 other variables represent the input variables $X_train$. 
%A revoir
Once a model is created it is tested with the second data set to test its accuracy.\\
\\
Among the input variables, one is categorical with four categories going from zero to four. 
%A discuter avant
This variable is only useful to determine if some other features are or not undefined. So the decision is made to not remove it.


%The ATLAS experiment at CERN provided simulated data used by physicists to optimize the analysis of the Higgs boson

\section{Models and Method}
%Describe your idea and how it was implemented to solve the problem. Survey the related work, giving credit where credit is due.


\section{Results}
%Show evidence to support your claims made in the introduction.

\section{Discussion}
%Discuss the strengths and weaknesses of your approach, based on the results. Point out the im- plications of your novel idea on the application concerned.

\section{Summary}
%Summarize your contributions in light of the new results.

\bibliographystyle{IEEEtran}
\bibliography{literature}

\end{document}
