\documentclass[10pt,conference,compsocconf]{IEEEtran}

\usepackage{hyperref}
\usepackage{graphicx}	% For figure environment


\begin{document}
\title{Machine Learning - Project 1}

\author{
  Julie Djeffal\\
  Lo\"{i}s Huguenin\\
  Fabien Zellweger\\
  \textit{Ecole Polythecnique Federale de Lausanne - Switzerland}
}

\maketitle

\begin{abstract}
  A critical part of scientific discovery is the
  communication of research findings to peers or the general public.
  Mastery of the process of scientific communication improves the
  visibility and impact of research. While this guide is a necessary
  tool for learning how to write in a manner suitable for publication
  at a scientific venue, it is by no means sufficient, on its own, to
  make its reader an accomplished writer. 
  This guide should be a starting point for further development of 
  writing skills.
\end{abstract}

\section{Introduction}


\section{The Structure of a Paper}



\section{Tips for Good Writing}



\end{document}
